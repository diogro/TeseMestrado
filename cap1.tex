% Faz com que o inicio do capítulo sempre seja uma página ímpar
\cleardoublepage
% Inclui o cabeçalho definido no meta.tex
\pagestyle{fancy}
% Números das páginas em arábicos
\pagenumbering{arabic}

\chapter{Introdução}\label{intro}

\section{Modularidade}\label{intro:modularidade}

Na imensa maioria dos organismos, conseguimos identificar partes relativamente discretas
e separadas, frequentemente envolvidas no desempenho de alguma função.
Em organismos unicelulares podemos distinguir organelas desempenhando
funções específicas, bem como regiões internas ou na membrana responsáveis por
processos distintos.
Já nos multicelulares, tipos celulares são organizados em tecidos espacialmente
separados, formando órgãos de funções distintas, que por sua vez são
organizados em sistemas responsáveis por funções distintas.
Modularidade se refere a esse padrão de organização dos seres vivos, onde
algumas partes são mais relacionadas entre si do que com outras partes
do mesmo organismo.
Podemos descrever, e entender, a organização entre partes
constituintes dos organismos através das relações entre elas, sendo cada
tipo de relação adequada a um nível de complexidade ou organização.
As partes do organismo as quais nos referimos podem ser as bases de uma
molécula de RNA \citep{Ancel2000}, genes \citep{Costanzo2010}, proteínas \citep{Han2004},
caracteres morfológicos como peso, altura ou marcos anatômicos
\citep{Klingenberg2008, Porto2009, Marroig2009}.
Essas relações podem ser medidas de diversas formas, como interação física
entre proteínas, padrões de expressão conjunta entre genes, ou, como é o caso
no presente trabalho, correlação entre caracteres quantitativos. 
Esse grupo de características muito relacionadas entre si constituem um
módulo, como esquematizado na figura \ref{modulos}. 
Módulos, então, são caracterizados por uma alta conectividade interna e
relativa independência de outros módulos.

A formação desses módulos reflete uma complexa interação entre genoma,
desenvolvimento e ambiente, e tem consequências diversas para o
funcionamento e evolução dos seres vivos.
Talvez o nível funcional seja o ponto de partida mais simples para
entender a origem da organização modular.
Interação com o ambiente e a necessidade de realizar determinada função
provoca respostas evolutivas nas populações, privilegiando o surgimento
de estruturas e sistemas adaptados para realizar essas funções.
Para que essas estruturas possam se adequar às diferentes pressões
evolutivas que cada parte do organismo está sujeita, uma relativa
independência entre elas se estabelece.
Essa independência se estabelece em vários níveis.
Através de experimentos controlados de cruzamento e uso de marcadores
genéticos, é possível mapear regiões no genoma envolvidas com
determinação de características macroscópicas nos organismos.
Com essas metodologias foi verificado que loci tendem a influenciar
regiões discretas, sendo poucos os genes com efeitos entre regiões
funcionais diferentes \citep{Cheverud1997}.
Genes que influenciam mais de uma característica de um indivíduo são
denominados pleiotrópicos.
Ou seja, seleção privilegia pleiotropia restrita a regiões funcionais
distintas \citep{Cheverud1984}.
Além disso, as estruturas e processos ontogenéticos responsáveis pela
formação dos indivíduos apresentam uma independência relativa, sendo
possível alterar parte do processo sem causar alterações ou danos
generalizados ao organismo.
O outro lado da moeda são os processos de integração, resultantes da
necessidade do organismo de manter a coerência interna dos seus módulos
e do indivíduo como um todo. 
Integração leva a aumento de relações e coesão entre as características
do organismo.
Evolutivamente, integração e modularidade permitem que grupos de
características funcionalmente ou ontogeneticamente ligadas se
modifiquem de forma harmoniosa; e que características em módulos
diferentes possam se alterar de forma independente.

\begin{figure}[h!]
  \centering
  \includegraphics[width=100mm]{figuras/modulos.png}
  \caption{Representação esquemática da organização modular dos seres
  vivos. As setes representam qualquer tipo de relação entre as partes
  de um indivíduo. Adaptado de \cite{Klingenberg2008}}
  \label{modulos}
\end{figure}

Em caracteres quantitativos, descritos em detalhe na seção
\ref{intro:genquant}, a expressão de todos os processos genéticos,
ontogenéticos e ambientais que formam a organização modular do organismo se
dá na forma de covariação (ou correlação) entre os caracteres.
Ou seja, a tensão entre os processos de modularidade e integração em todos os
níveis de organização do indivíduo podem ser percebidos na estrutura de
covariação final da população \citep{Klingenberg2008}.
Módulos caracterizados por alta correlação entre caracteres dentro do
módulo e baixa correlação entre caracteres de módulos diferentes são
chamados de módulos variacionais \citep{Wagner2007}.
Estes são frequentemente identificados como relacionados a funções
específicas, como mastigação, inserção muscular ou proteção
\citep{Cheverud1997}. 
Novamente, seleção privilegia interações pleiotrópicas e ontogenéticas restritas
a características ligadas a uma função comum, que resulta em
covariação entre essas características na população.


\section{Genética Quantitativa e Modularidade}\label{intro:genquant}

O estudo de caracteres quantitativos se baseia amplamente na teoria da
genética quantitativa. 
A genética quantitativa estuda a evolução de características contínuas
nos indivíduos de uma população, como peso, altura ou taxas de
crescimento \citep{Falconer1996}.
A partir desse formalismo, originário da criação humana de animais e plantas, 
podemos prever como a média e distribuição de
características fenotípicas varia em uma população de uma geração para
outra.

Caráteres contínuos são determinados geneticamente por muitos loci,
sendo portanto denominados caráteres multi gênicos.
A influência exata de cada um desses loci no fenótipo do individuo é
desconhecida, porém a combinação de todos esses efeitos e dos efeitos
ontogenéticos e ambientais resultam no valor macroscópico observado.
Em \cite{Crow1964} e \cite{Kimura1965} os autores descrevem um modelo
para os efeitos alélicos agindo sobre caracteres contínuos.
Neste modelo, chamado modelo do continuo de alelos, cada loci em
principio pode ter qualquer efeito sobre o tamanho de um caráter, e
qualquer um desses efeitos pode ser atingido via mutação.
A enorme variabilidade possível em sequencias genéticas e o número de
loci controlando cada caráter garante que essa suposição seja verossímil.
Nesse contexto o valor fenotípico de um caráter $p$ é dado pela soma dos
efeitos genéticos ($g$) e ambientais ($e$) atuando sobre ele:

\begin{equation}
    p = g + e
\end{equation}

A variação total de uma dada característica é resumida na sua variância
fenotípica, denominada $V_P$.
Esse valor de variância total é fruto da variação genética ($V_G$)
presente na população e da variação devido ao ambiente ao qual a população foi
sujeitada ($V_E$), além da interação entre elas. Como as variâncias são
aditivas, podemos escrever:

\begin{equation}
    V_P = V_G + V_E + V_{G \cdot E}
\end{equation}

Em organismos com sistemas de desenvolvimento estáveis, como é o caso
dos mamíferos, podemos ignorar o termo de interação em genótipo e
ambiente ($V_{G \cdot E} = 0$). Podemos continuar particionando as
variâncias. 
A variância genética é resultado de toda a variação devido a alelos sem
dominância (aditivos) e alelos com dominância.
Ou seja:

\begin{equation}
    V_G = V_A + V_D 
\end{equation}

O termo aditivo, $V_A$, representa a porção da variância genética
responsável pela herança por parentesco.
Isso se reflete na equação de resposta à seleção univariada.
A pressão seletiva sobre alguma característica fenotípica pode ser descrita
pela diferença na média da população parental antes ($\overline p$) e depois
($\overline p^*$) do evento de seleção.
Essa quantidade é denominada diferencial de seleção e é representada por
$S$. 
Esse tipo de seleção, que afeta média de um caráter na população, é
chamado de seleção direcional.
A resposta observada na média na geração seguinte ($\Delta z$) é dada
por:

\begin{equation}
    \Delta z = \frac{V_A}{V_P} (\overline p^* - \overline p) = h^2S
\end{equation}

Ou seja, a parcela da variância fenotípica devido à variação genética
aditiva define o quão eficiente a população será na sua resposta à
seleção. Essa parcela é denominada herdabilidade, representada por
$h^2$.
Mas essa equação ignora o efeito de seleção indireta em outras
características do indivíduo. Para levar em conta esses efeitos, devemos
lançar mão de um formalismo multivariado. 

A equação multivariada de resposta à seleção,
proposta por \cite{Lande1979}, permite ligar a mudança na média de um
conjunto de caracteres ($\Delta z$) ao diferencial de seleção imposto à população
($S$) e à suas matrizes de covariação genética aditiva e fenotípica (G e P). 

\begin{equation}
    \Delta z = GP^{-1}S
\end{equation}
 
$\Delta z$ e $S$ agora são vetores de mesma dimensão que o número
de características consideradas na analise, assim como as matrizes G e
P, que são os análogos multivariados das quantidades $V_A$ e $V_P$.
O caso multivariado apresenta características qualitativamente
diferentes do univariado.
Como os caracteres são ligados por efeitos genéticos correlacionados,
representados pela matriz G, a seleção direcional não é capaz de atuar de
forma isolada em cada caráter.
Seleção direcional, ainda que restrita a um caráter, provoca respostas
evolutivas em todos os caracteres que covariam com o caráter sobre
seleção.

Utilizando o conceito de superfície de seleção, proposto por
\cite{Wright1932}, podemos interpretar o termo de seleção $P^{-1}S$ de
forma geométrica. 
A superfície de seleção associa cada valor fenotípico para todos os
caracteres a um valor de aptidão ou fitness.
Caso essa superfície seja gaussiana, \cite{Lande1983} mostraram que 
$P^{-1}S$ é equivalente ao gradiente da média da superfície de seleção
na população. Por esse motivo, a quantidade  $P^{-1}S$ é chamada
de gradiente de seleção e representada pelo vetor $\beta$. 
Nas próximas seções vamos nos referir também a seleção estabilizadora.
Esse tipo de seleção age sobre a distribuição de um caráter, diminuindo
sua variabilidade, mas não sobre a média do caráter.
Com respeito à superfície de seleção, a seleção estabilizadora é
resultado da curvartura da supefície, ou da sua Hessiana
\citep{Lande1983}.
No caso de seleção direcional, podemos escrever a equação de resposta a seleção como:

\begin{equation}
    \Delta z = GP^{-1}S = G\beta
\end{equation}

A evolução dos caracteres por seleção direcional toma a forma de uma
subida de gradiente da superfície de seleção, alterado pela estrutura de
covariação genética, representada pela matriz G.

\section{Evolução da matriz G e a resposta à seleção}\label{intro:matG}

Na seção \ref{intro:modularidade}, nós descrevemos como os processos
seletivos interagem com a ontogenia e estrutura genética das populações
gerando modularidade.
Dissemos também que a modularidade presente nos processos internos dos
organismos se manifesta na população na sua estrutura da covariação.
Essa estrutura de covariação é representada pela matriz G.
Ela descreve como a variação aditiva, herdável, da população está particionada entre os
seus caracteres, e como estes se relacionam entre si.
Caracteres ligados por efeitos pleiotrópicos e ontogenéticos irão
apresentar covariação alta, enquanto caracteres independentes, não
relacionados, irão apresentar covariação baixa.
Como a população responde à seleção direcional, portanto, depende da sua estrutura
de modularidade e de como os caracteres estão relacionados entre si.

A equação de \cite{Lande1979} também pode ser invertida e usada de forma
retrospectiva.
Ou seja, partindo da matriz G e da diferença observada entre as médias
antes de depois do evento de seleção, podemos inferir qual foi o
gradiente de seleção que gerou essa resposta.
Mais claramente:

\begin{equation}
    \beta = G^{-1}\Delta z
\end{equation}

Além disso, esse mesmo raciocínio pode ser usado para investigar padrões
microevolutivos de resposta ao longo de muitas gerações
\citep[veja também a equação \ref{betatotal}]{Lande1983, Marroig2004, Marroig2005}. 

Porém, nossa habilidade de inferir a resposta evolutiva a muitas
gerações de seleção depende fundamentalmente da constância da matriz G
ao longo de muitas gerações, e a teoria de Lande não explora
detalhadamente a dinâmica da própria matriz G, assumindo implicitamente
nas equações sua constância (ao menos em escalas de tempo pequenas) ao
supor que a distribuição das características na população é sempre
gaussiana e portanto não afetada pela seleção direcional
\citep{Barton1987}. 
Essa suposição simplificadora de Lande se baseia em um resultado
derivado do modelo do continuo de alelos, demonstrado por
\cite{Kimura1965}: sobre seleção estabilizadora, a
distribuição de equilíbrio dos efeitos alélicos sobre um determinado
caráter é gaussiana, supondo que os efeitos mutacionais sejam pequenos
frente à variação existente na população. 
Ou seja, assumindo que os efeitos alélicos sejam efetivamente
gaussianos, a dinâmica da média da população pode ser completamente
descrita apenas pela média e pela matriz de covariação desses efeitos
\citep{Barton1987}. 

Notamos então que uma série de suposições não triviais devem ser feitas
para que esse modelo gaussiano quadrático de Lande seja teoricamente
plausível. 
Esses problemas foram apresentados ao longo da década de 80 por uma
série de artigos, principalmente e pioneiramente por \cite{Turelli1984,
Turelli1985, Turelli1986, Barton1987, Barton1989}. 
Talvez a questão mais importante seja a suposição de que os alelos
tenham distribuição gaussiana proposta por \cite{Kimura1965}. 
Para que isso seja realista, a taxa de mutação por locus deve ser
suficientemente alta para manter a distribuição de alelos por locus
gaussiana frente à deriva e seleção estabilizadora, que removem
variabilidade \citep{Falconer1996}. 
Além disso, podemos nos perguntar se mutações novas realmente tem efeitos
pequenos em relação a variância original da população nas direções
morfológicas afetadas, ou se novas mutações introduzem variabilidade
maior que a originalmente presente na população. 
O modelo proposto por \cite{Turelli1984} utiliza taxas de mutação
baixas e efeitos mutacionais altos, mas ainda mantendo o espaço de
alelos contínuo. 
Neste modelo, a maior parte da variação fenotípica acaba sendo atribuída
a alelos raros, de baixa frequência, mas com efeitos aditivos grandes, e
não a vários alelos polimórficos de efeitos pequenos e distribuição
gaussiana. 
Nessas condições, a matriz G não seria constante ao longo do tempo e
fatores de correção deveriam ser acrescidos à equação de resposta de
Lande para descrições macro evolutivas. 
Esses fatores dependem do desvio por geração da matriz G em relação à
matriz G média no período de seleção \citep{Jones2004}. 
A estimativa desse ``efeito Turelli'' é difícil de ser realizada
praticamente. 

Em \cite{Barton1987}, os autores, interessados no problema da dinâmica
das variâncias de forma geral, apresentam um formalismo relativamente
completo para lidar com o problema da evolução de todos os momentos da
distribuição fenotípica de um caráter quantitativo, abordando tanto a
aproximação gaussiana quanto a aproximação de alelos raros, proposta por
\cite{Turelli1984}. 
O procedimento seguido pelos autores é escrever a dinâmica de mudança de
todos o momentos da distribuição de efeitos alélicos da população
sujeita a uma pressão de seleção na forma de uma subida de gradiente
imposta por uma superfície de seleção arbritária (mas com todas as
derivadas parciais bem definidas no espaço dos momentos)
\citep{Arnold2001a}. 
Isso permite abarcar o formalismo de Lande, tomando apenas os dois
primeiros momentos (média e variância), supondo o segundo momento fixo e
desprezando os momentos mais altos; ou incluir mudanças até uma certa
ordem e obter os resultados de Turelli. 

% TODO Continuar Turelli Barton?

Dada a complexidade do problema da dinâmica das variâncias em um caso
multivariado, o estudo da estabilidade da matriz G e suas consequências
para o estudo da evolução de caracteres quantitativos se tornou um
problema eminentemente experimental, sejam esses experimentos conduzidos
de forma retrospectiva, amostrando a diversidade dos seres vivos, seja
em experimentos manipulativos feitos com colônias de animais em
cativeiro ou, como é o caso do presente trabalho, por via de simulações
computacionais. 
Vamos agora revisar as principais abordagens computacionais. 

A sequência de artigos \cite{Jones2003, Jones2004, Jones2007} foi
pioneira nos estudos computacionais de estabilidade da matriz G num
contexto multivariado moderno, abordando o tema da evolução de dois
caracteres correlacionados em diferentes situações. 
O estilo de simulação apresentado em \cite{Jones2003} e utilizado em
todos os artigos subsequentes serviu de ponto de partida para nossas
simulações e será descrito em detalhes nas próximas seções. 

O problema mais simples, abordado em \cite{Jones2003}, consiste no
estudo de estabilidade da matriz G associada a dois caracteres
quantitativos com interações pleiotrópicas fixas em uma população finita
sofrendo mutação, deriva e seleção estabilizadora gaussiana. 
Os principais parâmetros que influem na estabilidade da matriz G nessas
simulações são o padrão de mutação pleiotrópico e a superfície de
seleção. 
Ambos são descritos por matrizes de covariância. 
Os valores na diagonal dessas matrizes representam a intensidade das
mutações e a intensidade da seleção estabilizadora em cada caráter
separadamente. 
Como o problema é bidimensional, essas matrizes tem apenas um valor fora
da diagonal. 
Esse valor é apresentado como uma correlação de mutação no caso da
matriz mutacional ($r_\mu$) e uma correlação de seleção na matriz que
descreve a superfície de aptidão ou {\it fitness} ($r_\omega$). 
O valor de $r_\mu$ define o quão correlacionados serão os valores das
mutações que afetam ambos os caracteres. 
Quanto mais alto, mais correlacionados eles serão, com sua magnitude
dada pelos valores diagonais da matriz de mutação. 
Já o valor de $r_\omega$ define o quão forte é a seleção estabilizadora
no sentido de manter os dois caracteres correlacionados. 
Nas condições das simulações, o fator mais importante para a
estabilidade da matriz G foi a existência de mutações correlacionadas. 
Quando $r_\mu$ e $r_\omega$ são similares, a matriz tende a ser bastante
estável. 
Além disso, tamanhos populacionais grandes também contribuem para a
estabilidade da matriz. 
Um resultado interessante é que existem tipos diferentes de estabilidade
para matrizes de covariância. 
Por exemplo: uma matriz pode manter seus autovalores estáveis, não
sofrendo alterações na quantidade de variância genética disponível, mas
ter autovetores variáveis, alterando assim em qual direção do
morfoespaço a variabilidade populacional está disponível para responder
a seleção. 
Os autores ressaltam que matrizes com um autovalor dominante tendem a
ser mais estáveis. 
Ou seja, correlação alta entre os dois caracteres leva à estabilidade da
matriz G. 
Isso pode ser interpretado como um primeiro indício para a importância
da modularidade (veja a seguir), apesar de ser difícil definir
modularidade de maneira convincente com apenas dois caracteres. 
Em resumo, esse artigo mostra que é possível observarmos matrizes
estáveis em condições plausivelmente naturais, e dá o primeiro passo
para a quantificação dessas condições. 

O próximo passo foi a inclusão de seleção direcional em \cite{Jones2004}. 
O estilo de simulação é basicamente idêntico ao artigo anterior, mas
agora a posição do pico adaptativo gaussiano é variado, alterando assim
a média da população e criando uma seleção direcional. 
Além de permitir abordar a questão da estabilidade sob novas condições,
essa nova simulação permite também explorar a reconstrução da seleção
direcional via equação de \cite{Lande1979} e quantificar o efeito de
Turelli devido a flutuações na matriz G. 
Para entender o efeito Turelli, vamos explorar como a extensão da
equação de Lande deve ser feita para abarcar a mudança evolutiva ao
longo de várias gerações. 
Começamos com a equação de resposta à seleção nas médias dos caracteres
($\overline {z}$) para uma única geração de uma população sujeita ao
gradiente de seleção $\beta_t$:

\begin{equation}
\Delta \overline {z} = \overline {z}_{t+1}-\overline {z}_{t}=G\beta_t
\end{equation}

Com essa equação, e de posse das médias antes e depois da seleção e da
matriz G dessa população, podemos estimar o valor de $\beta_t$. 
Além disso, podemos pensar em definir um gradiente médio $\beta_T$  ao
longo de várias gerações de seleção \citep{Lande1979}, como:

\begin{equation}
\beta_{T}\equiv \sum _{t=0}^{T-1} \beta_t =  \overline {G}^{-1}\Delta \overline {z}_T 
\label{betatotal}
\end{equation}

ou seja, a soma dos gradientes individuais, onde $\overline {G}$
representa a matriz G média e $\Delta \overline {z}_T$ a mudança global
na média dos caracteres. 
\cite{Turelli1988} mostrou que essa equação se torna incorreta caso a
flutuação da matriz G em torno da média seja grande entre uma geração e
outra. 
Isso fica claro expandindo essa equação de forma a mostrar os termos de
flutuação:

\begin{equation}
   \beta_T = \overline {G}^{-1} \left[ \Delta \overline {z}_T - \sum_{t=0}^{T-1} (G_t - \overline {G}) \beta_t\right]
\end{equation}

Se o termo de Turelli (  $\sum_{t=0}^{T-1} (G_t - \overline {G})
\beta_t$ ) for grande, a estimativa de $\beta_T$ será precária. 
É necessário então avaliar se a estabilidade da matriz G é suficiente
para manter esse termo pequeno o bastante para possibilitar a
estimativa de gradientes de seleção realistas. 
Para isso, foram realizadas simulações com 3 tipos de movimentos do pico
adaptativo: ao longo da direção de aumento de um caráter, mantendo o outro
estável; aumentando simultaneamente os dois caracteres, na direção usualmente
descrita como direção de tamanho, por representar aumento simultâneo de
todas os caracteres do indivíduo \citep{Marroig2005}; e na direção de
aumento de um caráter e diminuição do outro. 
Estas situações são representadas pelos símbolos $\rightarrow$,
$\nearrow$ e $\searrow$, respectivamente, fazendo uma alusão à
representação bidimensional cartesiana dos caracteres. 
Elas são interessantes por representarem tipos de seleção que interagem
de forma diferente com a presença da correlação mutacional. 
Quando $r_\mu$ for positivo, a seleção $\nearrow$ está alinhada com o
eixo de maior variação da matriz M. 
Esta direção é denominada eixo de menor resistência evolutiva
\citep{Schluter1996}, e  representa a direção do morfoespaço com maior
quantidade de variação para responder a seleção natural, sendo portanto
o primeiro componente principal da matriz de covariação genética. 
Já as outras duas direções de movimento do pico se dão em eixos
diferentes, com o $\searrow$ sendo o mais antagônico ao eixo de menor
resistência. 
Sob essas condições, foi verificado que todos os fatores promotores de
estabilidade no caso do ótimo fenotípico fixo continuam válidos. 
Além disso, com $r_\mu$ e $r_\omega$ de mesmo sinal, ou seja, com
alinhamento da matriz mutacional com a matriz de seleção, o movimento
do pico adaptativo na direção da linha de menor resistência evolutiva
cria uma matriz G ainda mais estável. 
Em contrapartida, seleção direcional em outras direções tende a
desestabilizar a matriz e, para algumas combinações de parâmetros, criar
um efeito de maladatação permanente, onde a média da população não
coincide com o ótimo da superfície de seleção. 
Quanto ao efeito Turelli, os resultados sugerem que mesmo em populações
relativamente pequenas ($N_e=342$) a magnitude do efeito é entorno de
5-7\% na norma de $\beta_T$, diminuindo para até 1-2\% em populações
maiores ($N_e=2731$). 
O efeito Turelli então pode ser pequeno em situações naturais, mas
sempre existem condições onde ele pode ser relevante, como quando existe
correlação entre $\beta_t$ e $G_t$ por muitas gerações seguidas. 
A mensagem então é que, apesar de existirem situações onde o efeito
Turelli é importante, não é de se esperar que estas sejam a regra. 
Apesar disso, os autores alertam que na prática estimativas fidedignas
de $\overline {G}$ são bastante difíceis de serem obtidas, tanto pelas
flutuações entre gerações quanto pela dificuldade em se obter tamanhos
amostrais adequados para estimativas de matriz G \citep{Marroig2011b}.

O mais recente artigo nessa sequência \citep{Jones2007} trata da
evolução da própria matriz mutacional e das consequências das mudanças
nessa matriz para o confronto entre as visões de Lande (gaussiana) e de
Turelli (alelos raros) na questão das distribuições dos efeito alélicos
atuando em caracteres quantitativos. 
Para tal foi definido um terceiro caráter que controla o valor de
$r_\mu$ e está sujeito a mudanças via mutação, de forma análoga aos
outros caracteres (veja seção \ref{cap2:mem:ModelM} para mais detalhes). 
Isso permitiu aos autores estudar como o padrão de pleiotropia e
epistasia, representado na matriz mutacional, pode variar e como isso
afeta o fenótipo e o genótipo de uma população sujeita a seleção
estabilizadora gaussiana, mutação e deriva. 
Essa é a abordagem mais interessante para nossos objetivos neste
trabalho, que visa entender em quais condições sistemas modulares podem
evoluir. 
Para isso uma estrutura pleiotrópica plástica é imprescindível
\citep{Wagner1996, Pavlicev2011a}. 
Já vimos que matrizes mutacionais e seletivas alinhadas promovem
estabilidade da matriz G. 
A pergunta que mais nos interessa no artigo de \cite{Jones2007} é a que
se refere ao tipo de seleção que atua sobre o valor de $r_\mu$: seria a
seleção estabilizadora gaussiana, descrita por um valor de correlação
$r_\omega$, capaz de moldar o valor de $r_\mu$? 
Ou seja, será que ocorre o alinhamento da matriz mutacional (e portanto,
da matriz G)  com a matriz da superfície de seleção? 
A seleção atua somente sobre caracteres fenotípicos, logo sua influência
sobre a matriz mutacional é indireta. 
Será que essa força de seleção indireta é suficiente para superar
flutuações aleatórias na matriz mutacional e proporcionar seu
alinhamento? Apesar de fraca, a seleção indireta sobre a matriz
mutacional se mostrou capaz de promover alinhamento com a superfície
adaptativa, e portanto promover uma estabilidade maior da matriz G
\citep{Jones2007}. 
Esse resultado é interessante por misturar a atuação de duas forças
seletivas importantes: seleção estabilizadora clássica, ambiental,
externa ao organismo; e restrições internas aos organismos,
representadas pelos seu padrão de pleiotropia e codificados aqui na
matriz mutacional. 
Dois tipos de restrições diferentes se influenciando mutuamente e
afetando o padrão de covariação e expressão fenotípica dos sistemas
biológicos. 
Além desse resultado, os autores mostram que nem o modelo gaussiano nem
o de alelos raros são capazes de explicar totalmente os resultados,
levando a uma visão intermediaria para a real distribuição dos efeitos
alélicos nas populações. 

\section{Consequências Evolutivas}\label{intro:consequencias}

A interação entre restrições externas e internas pode ser fundamental
para entendermos padrões naturais de covariação. 
É devido a essa combinação das restrições internas, derivadas do
desenvolvimento, e de restrições funcionais, seletivas, que o padrão de
integração e modularidade se estabelesse e se manisfesta na estrutura de
covariação das populações.
Vamos abordar agora evidências empíricas ao problema da estabilidade da
matriz G e suas consequências evolutivas, tema de trabalho há mais de 10
anos do Laboratório de Evolução de Mamíferos. 

Em \cite{Marroig2001}, os autores utilizam o clado Platyrrhini, de
primatas neotropicais como modelo, obtendo estimativas e comparando de
forma filogeneticamente estruturada os 16 gêneros, quanto às suas
matrizes de covariação fenotípica para 39 caracteres cranianos. 
Essa comparação revelou similaridades altas entre todos os gêneros. 
Além disso, os níveis de similaridade presentes eram pouco
correlacionadas com a filogenia e sim com fatores ambientais, como a
dieta de cada espécie. 
Isso reforça a ideia de alinhamento das matrizes de covariação com o
ambiente seletivo ao qual cada população é sujeita. 
Além disso, a similaridade alta dos padrões, mesmo num clado com 30
milhões de anos de divergência e com variação morfológica notável,
aponta para uma estabilidade marcante da matriz G. 
Além da comparação de padrões, em \cite{Marroig2005, Marroig2010} os
autores abordam a influência desses padrões genéticos estáveis na
evolução dos caracteres fenotípicos dentro dos primatas neotropicais. 
Segundo eles, a influência de restrições genéticas é clara e deve ser
levada em conta nas tentativas de entender os processos evolutivos que
formaram os grupos atuais. 

Expandindo o padrão de estabilidade da matriz G, em \cite{Porto2009} e
\cite{Marroig2009} comparações entre matrizes obtidas para todas as
ordens de mamíferos confirmam novamente o padrão de estabilidade da matriz
G, além de evidenciar a importância da modularidade para compreensão das
respostas evolutivas possíveis e observadas. 

A relação entre a organização modular e a evolução pode ser descrita por
algumas estatísticas associadas ao padrão de covariação das populações
\citep{Hansen2008}. 
Uma dessas estatísticas é a flexibilidade evolutiva, definida como a
correlação média entre a resposta observada numa população e o gradiente
de seleção que gerou essa resposta \citep{Marroig2009}. 
Quanto mais alta a flexibilidade, mais capaz a população é de responder
na direção morfológica privilegiada pela seleção natural. 
A evolvabilidade, complementar à flexibilidade, mede o quão grande foi a
mudança na direção do gradiente de seleção, ou, mais claramente, mede
quanta variação havia disponível na direção privilégiada pela seleção. 
Essas estatísticas permitem descrever a capacidade de mudanças evolutivas
nas populações, e podem ser usadas de forma preditiva
\citep[veja, por exemplo,][]{Marroig2010}. 
Modularidade é frequentemente associada à evolvabilidade de uma
população, pois uma estrutura modular permite que partes do organismo sejam
modificadas pela ação da seleção natural sem simultaneamente causar
alterações deletérias a alguma outra parte não envolvida no processo de
seleção em questão. 
Apesar disso, um balanço entre modularidade e integração deve se
estabelecer no organismo para que este forme um todo coeso. 
Um indivíduo constituído de caracteres completamente independentes poderia
em teoria ter evolvabilidade e flexibilidade máximas, mas a ausência de
interrelação entre suas partes torna tal organismo impossível. 

\section{Evolução da Modularidade}\label{intro:evolucao}

Além de interferir em como os organismos respondem à seleção natural, a
modularidade também é fruto da seleção \citep{Wagner1996, Wagner2007}. 
Como o padrão modular é praticamente ubíquo em sistemas biológicos, o
problema de como ele surgiu é difícil de ser abordado de forma
experimental. 
Algumas evidências de condições para emergência da modularidade são
encontradas em experimentos com sistemas muito simples, como cadeias de
RNA \citep{Ancel2000} ou redes metabólicas simplificadas
\citep{Espinosa-Soto2010}. 
Em ambos os casos, a seleção funcional é fundamental para a emergência
de modularidade no nível estudado.

Em sistemas quantitativos, \cite{Pavlicev2010} mostraram que em um
sistema contando com variação epistática no padrão de correlação de dois
caracteres, a seleção direcional é capaz de promover a fixação do alelo
responsável pela manutenção de alta correlação entre os caracteres sobre
seleção direcional simultânea  (seleção do tipo $\nearrow$ em
\cite{Jones2004}). 
Esse artigo, porém, faz uso direto da equação de seleção de
\cite{Lande1979} e não simula explicitamente os alelos responsáveis pelo
fenótipo sobre seleção direcional, se restringindo aos alelos
responsáveis pela correlação entre os caracteres na matriz G. 
A existência desses alelos responsáveis pela covariação entre caracteres foi
comprovada experimentalmente, e estes foram chamados de loci
quantitativos de relação, ou rQTLs \citep{Pavlicev2008a}. 
De qualquer forma, essa é mais uma evidência do poder da seleção
direcional de criar associação entre caracteres e portanto promover a
modularidade em condições naturais. 

Nosso objetivo neste trabalho foi estabelecer uma técnica de modelagem
que permita simular a evolução de caracteres quantitativos controlados
por efeitos aditivos expressos em populações sujeitas a seleção natural. 
Com isso esperamos adquirir um melhor entendimento dos padrões de
diversificação e integração morfológica observados em populações
naturais, além de testar a viabilidade de possíveis cenários
evolutivos que moldaram essas populações. 
No capítulo \ref{cap2} vamos apresentar dois modelos possíveis para a
simulação de caracteres quantitativos. 
No capítulo \ref{cap3} vamos estudar o comportamento de populações
simuladas pelos dois modelos e avaliar sua capacidade de reproduzir
padrões modulares encontrados na natureza sobre diferentes condições de
seleção e parâmetros dos modelos. 
No capitulo \ref{cap4} vamos utilizar uma determinada combinação de
parâmetros para estudar possíveis cenários de evolução de padrões
modulares. 
