\pagestyle{empty}
\cleardoublepage
\pagestyle{fancy}
\chapter{Modelagem}\label{cap2}

\section{Introdução}\label{cap2:intro}

O primeiro passo do projeto foi escolher o modelo a ser implementado.
Foi necessário encontrar uma técnica computacional que permitisse
simular explicitamente cada individuo de uma população com alelos de efeitos
contínuos em traços morfológicos. 
Esses traços seriam então apresentados à seleção natural.
As populações são diploides, hermafroditas e se reproduzem sexuadamente,
com gerações discretas e tamanho populacional fixo.
Existem algumas possibilidades de como implementar as mutações e
a transição de efeitos alélicos para traços fenotípicos.

\section{Modelos}\label{cap2:mem}

\subsection{Modelo com Matriz $M$}\label{cap2:mem:ModelM}

O modelo usado em \cite{Jones2003, Jones2004, Jones2007} se baseia em um
conjunto de alelos pleiotrópicos.
Nesse sistema, todos os alelos afetam todos os traços.
Num sistema com $p$ traços, podemos pensar em cada um
dos $m$ alelos como sendo um vetor em $\mathbb{R}^p$.
O conjunto de todos os alelos $x^i$ seria escrito como:
\begin{equation}
x^i = ( x^i_1, x^i_2,\ldots, x^i_p), i \in \{1,\ldots, m\}, x^i \in \mathbb{R}^p
\end{equation}
onde cada um dos $x^i_j$ com $j \in \{1,\ldots, p\}$ é um número real
representando o efeito aditivo do alelo $i$ no traço $j$
Cada um dos
$p$ traços é formado pela soma dos efeitos alélicos de cada um dos
$m$ alelos agindo sobre esse traço, além de um efeito ambiental $e$, com
distribuição normal e média zero.
Um traço $z_j$ é dado
por:
\begin{equation}
z_j = \sum_{i=1}^m x^i_j + e, z_j \in \mathbb{R}, e \sim N(0, 1)
\end{equation}

A seleção nesse modelo é gaussiana.
Logo, a aptidão do indivíduo com fenótipo $z$ é dada por:

\begin{equation}
W(z) = exp \left[-\frac{1}{2} ((z-\theta)^T \omega^{-1} (z-\theta))\right] 
\label{selecao}
\end{equation}

onde $\omega$ representa a matriz de covariação da superfície de seleção
gaussiana e $\theta$ o pico adaptativo, ou seja, o fenótipo com maior
aptidão.
Alterando $\theta$ podemos impor regimes de seleção direcional.
A cada geração a função $W(z)$ representa a probabilidade de reprodução
de um indivíduo com fenótipo $z$.
Quanto maior $W(z)$ em relação aos outros indivíduos, mais descendentes
o indivíduo $z$ terá na próxima geração.
A seleção estabilizadora é controlada pela matriz $\omega$, e
pode ser de dois tipos: correlacionada e não correlacionada.
A seleção estabilizadora mais simples, não correlacionada, é codificada
em uma matriz $\omega$ diagonal e não influencia a correlação entre os
traços, apenas sua média e variância.
Já a seleção estabilizadora correlacionada é dada por uma matriz
$\omega$ com valores não nulos fora da diagonal e afeta a correlação
entre os traços. 

As mutações nesse modelo são gaussianas e afetam um alelo por vez.
A frequência com que acontecem mutações é dada por uma probabilidade de
mutação por alelo por geração, definida {\it a priori}.
O valor de um alelo $x^i$ é alterado por uma variação $ \delta x$.
Após a mutação, $x^i$ assume um certo valor $x'^i$:

\begin{equation}
x'^i = x^i + \delta x = ( x^i_1 + \delta x_1, x^i_2 + \delta x_2,\ldots, x^i_p + \delta x_p)
\end{equation}

O valor de $\delta x$ é tomado de uma distribuição gaussiana com média
zero e matriz de covariância $M$, chamada matriz mutacional:

\begin{equation}
\delta x \sim N(0, M) = exp \left[-\frac{1}{2} (\delta x^T M^{-1} \delta x)\right] , \delta x \in \mathbb{R}^p
\end{equation}

A matriz $M$ resume todo o conjunto de interações pleiotrópicas em cada
individuo, definindo como os efeitos de cada alelo se correlacionam no
que diz respeito aos diferentes traços fenotípicos.
A diagonal da matriz $M$, ou seja, as variâncias mutacionais, que
definem a escala dos efeitos mutacionais, é definida {\it a priori}.
Já os valores fora da diagonal, as covariâncias mutacionais, podem ser
definidas a priori \citep{Jones2003, Jones2004} ou podem ser definidas
por um segundo conjunto de genes com seus próprios efeitos alélicos
aditivos \citep{Jones2007}.
Uma matriz mutacional variável permite explorar como mudanças nos
efeitos pleiotrópicos se comportam em diferentes condições.
A matriz mutacional é controlada por um conjunto de alelos mutacionais
$\mu^{ij}_k$ (com $k \in \{1,\ldots,m_\mu\}$), que codificam um valor de
correlação mutacional $r_\mu^{ij}$, que por sua vez são usado para gerar
o valor da covariação entre os traços $i$ e $j$ na matriz $M$:
\begin{equation}
M_{ij} = r_\mu^{ij} \sqrt {M_{ii}M_{jj}}
\end{equation}
com
\begin{equation}
r_\mu^{ij} = \Phi \left(\sum_{k=1}^{m_\mu} \mu^{ij}_k\right)
\end{equation}
Como todos os $r_\mu^{ij}$ são correlações, o valor deles deve ir de
$-1$ a $1$.
A função $\Phi$ faz esse papel, levando a soma de valores
aditivos dos $\mu^{ij}_k$ de valores reais a valores entre $-1$ e $1$.
Um exemplo de função $\Phi$, usado em \cite{Jones2007} é:

\begin{equation}
\Phi (x) = \frac{2e^{2x}}{1+e^{2x}} - 1
\end{equation}
% phi = plot ((2*exp(2*x))/(1 + exp(2*x))) 1

Essa função garante que os valores de $r_\mu^{ij}$ permaneçam em
intervalos possíveis para valores de correlação.

Quando temos apenas dois traços, a correção dada pela função $\Phi$ é
suficiente para garantir que a matriz $M$ seja bem definida.
Porém, quando vamos tratar de sistemas com mais caracteres precisamos
tomar cuidados adicionais.
Isso se deve ao fato da matriz $M$ ser uma matriz de covariância, e
portanto necessariamente positiva definida \citep{Anderson1984}.
Existem várias condições equivalentes para garantir que uma matriz seja
positiva definida, por exemplo ter todos os autovalores estritamente
positivos ou a existência da sua descomposição de Cholesky\footnote{A
   decomposição de Cholesky de uma matriz positiva definida simétrica
$C$ é dada por $C=LL^*$, onde $L$ é uma matriz triangular e o asterisco
representa a transposição complexa conjugada.}.
Na extensão para mais de dois traços, nós optamos por verificar se a
matriz $M$ é positiva definida aplicando a decomposição de Cholesky:
caso a decomposição existisse, ela pode ser usada para gerar os valores
de $\delta x$ de forma eficiente; caso a decomposição não existisse um
procedimento de extensão \citep{Marroig2011b} era realizado,
substituindo todos os autovalores não positivos pelo último positivo.
Esse procedimento garante uma matriz ainda próxima à matriz $M$
original, mas positiva definida.
A escolha desse procedimento é relativamente arbritária, mas
considerações de eficiência computacional são importantes, pois a
obtenção de autovalores é bastante custosa.

\subsection{Modelo com Matriz $B$}\label{cap2:mem:ModelB}

Além de testar o uso do modelo utilizando a matriz $M$ para mais de dois
traços, nós optamos por criar também um modelo próprio, inspirado pela
descrição feita em \cite{Wagner1984}.
O modelo descrido por Wagner também faz uso do valor dos efeitos
alélicos $x \in \mathbb{R}^m$.
Nesse caso, porém, a transição para o fenótipo $z \in \mathbb{R}^p$ é
feita via uma função ontogenética $f$

\begin{equation}
z = f(x), f:\mathbb{R}^m \rightarrow \mathbb{R}^{p}
\end{equation}

A função ontogenética resume todas as interações pleiotrópicas e
epistáticas, e pode ser altamente não linear e complexa.
Supondo que $f$ seja diferenciável, podemos aproximá-la por uma série de
Taylor:

\begin{equation}
z = f(x) = f(0) + x^T D[f(0)] + \frac{1}{2!} x^T D^2 [f(0)] x + \ldots
\end{equation}

Tomando $f(0) = 0$ e mantendo apenas o termo linear, podemos considerar
a aproximação linear da função $f$ como:

\begin{equation}
z = f(x) = D[f(0)]x = Bx + e
\label{matrizB}
\end{equation}

Onde $B$ é definida como uma ``matriz ontogenética'', que descreve de
forma simplificada como os efeitos aditivos são convertidos em um
fenótipo, resumindo, de forma binaria e linear, todos os efeitos
pleiotrópicos e epistáticos entre os alelos aditivos.
Como no modelo anterior, acrescentamos também um valor de ruido
ambiental $e$, com distribuição multivariada normal e média zero.
A princípio as entradas da matriz $B$ poderiam tomar qualquer valor
real, porém como simplificação nós optamos por simular a matriz $B$ como
sendo apenas uma matriz de influência, ou seja, no nosso modelo ela
codifica se um determinado gene tem seu valor aditivo adicionado a um
traço ou não.
Resumidamente, caso $B_{ij} = 1$ o traço $z_j$ tem seu valor
incrementado por $x_i$, caso contrário ($B_{ij} = 0$) não.
Nessas condições a equação \ref{matrizB} continua válida.
A razão entre a dimensão do vetor de valores aditivos $m$ e o número de
traços $p$ é relevante \citep{Wagner1984}, pois ela define a grosso modo
a quantidade de recombinação presente no modelo.
Quando maior for a razão $m/p$, mais
recombinação teremos, pois os alelos são herdados de forma independente.

Mutações são consideravelmente mais simples que no caso da
matriz $M$, porém são divididas em duas classes: as genéticas e as
ontogenéticas.
As genéticas afetam o valor de cada $x_i$ individualmente, e possuem uma
taxa de mutação $\mu$ por geração por loci definida {\it a priori}.
Uma mutação no loci $i$ altera o valor de $x_i$ por um valor real tomado
de uma distribuição normal com média zero e variância pré-definida.
Já as mutações ontogenéticas alteram a matriz $B$, afetando cada casela
da matriz com probabilidade $\mu_B$ por geração por casela.
Uma mutação na posição $ij$ altera o valor de $B_{ij}$ de $0$ para $1$
ou vice-versa.
Como esses dois tipos de mutação impõem dinâmicas temporais em escalas
diferentes para os dois níveis de complexidade, nós testamos razões
variadas para $\mu$ e $\mu_B$

A seleção nesse modelo toma a mesma forma que no caso anterior, expressa
na equação \ref{selecao}.
Novamente a seleção estabilizadora pode ser correlacionada ou não
correlacionada, dependendo da matriz $\omega$ e a seleção direcional é
aplicada utilizando um ótimo variável na superfície gaussiana.

\section{Parâmetros}

Alguns parâmetros foram mantidos constantes em todas as simulações.
Os valores da variância mutacional de todos os traços foi fixada em 0.5
no modelo com matriz $M$ e 0.2 no modelo com matriz $B$.
No caso do modelo com matriz $M$ o valor da variância das mutações no
alelos mutacionais foi de 0.05.
A variância dos efeitos ambientais foi fixada em 0.4 nas simulações com
matriz $M$ e 0.8 nas simulações com matriz $B$.
O valor da variância ambiental define a escala do modelo e é importante
na determinação de herdabilidade dos traços.
A taxa de mutação nos alelos aditivos ($\mu$) foi mantida em 0.0005 em
todas simulações.
Todos esses parâmetros foram escolhidos de forma a manter a
herdabilidade dos traços por volta de 0.6, um valor biologicamente
plausível e capaz de responder a seleção direcional de forma eficiente.

Nas simulações nós alteramos o tamanho populacional ($Ne$), o número de
traços ($p$), o número de alelos ($m$), a taxa de mutação nas caselas da
matriz $B$ ($\mu_B$), e o tipo de seleção (estabilizadora e direcional).
Alteramos também a intensidade da seleção direcional.
Para isso, modificamos o quanto o pico adaptativo era alterado para cada
traço a cada geração, de $0.01$ até $0.2$.
Esse deslocamento por traço é chamado de $\Delta_S$.
A variância fenotípica das populações em equilíbrio mutação-seleção
direcional era por volta de $1.6$.
Então, nossa seleção mais intensa implicava um movimento de pico de
cerca de $0.15$ desvios padrões por geração por traço ($\Delta_S=0.2$),
resultando num movimento total de cerca de 150 desvios padrões por
traço; enquanto a seleção mais fraca um movimento de $0.007$ desvios
padrões por geração por traço ($\Delta_S=0.01$), resultando num
movimento total de cerca de 7 desvios padrões.

\section{Estatísticas}

Quantificar a evolução das relações entre os traços quantificadas na
matriz de covariância é fundamental para a interpretação dos resultados
das simulações. 
Podemos simplesmente inspecionar visualmente as matrizes, mas medidas
quantitativas são importantes quando o volume de dados se torna
suficientemente grande. 
Para isso, vamos explorar algumas estatísticas descritivas já utilizadas em
estudos de biologia tradicionais e apresentar técnicas novas para a
quantificação de modularidade baseadas nas matrizes de correlação.

Quando conhecemos de antemão a estrutura modular apresentada na matriz
de correlação, podemos quantificar o nível de modularidade medindo
apenas a médias das correlação dentro de e entre módulos, quantidade
conhecida como razão das médias, ou AVG-Ratio do inglês {\it AVeraGe Ratio}. 
Certamente essa medida é informativa em quantificar o quão evidentes são
os módulos variacionais expressos na população por quantificar a
diferença no nível de correlação dentre e entre módulos. 
Apesar disso, essa medida exige conhecimento prévio da estrutura de
modularidade presente na população. 
Esse certamente não é o caso quando amostrando uma população natural.
Para sanar esse problema, propomos também uma estatística de
modularidade inspirada em técnicas de detecção de comunidades em redes
\citep{Newman2006,Newman2006a,Reichardt2006}.
O princípio dessa técnica é alocar cada traço em um módulo de modo que
algum parâmetro de modularidade seja máximo.
Para isso, definimos a modularidade $L$ de uma matriz de correlação $A$
como sendo:

\begin{equation}
   L = \sum_{i \neq j} \left[ A_{ij} - \frac{k_ik_j}{2m} \right] \delta(g_i, g_j)
\end{equation}

Os termos $g_i$ e $g_j$ representam a partição dos traços, ou seja, em
qual grupo os traços $i$ e $j$ se encontram. 
A função $\delta(\cdot,\cdot)$ é a função delta de Kronecker, ou seja:

\begin{equation}
   \delta (x,y) = \left \{ 
      \begin{array}{rl}
          1 & \text{se } x = y\\
          0 & \text{se } x \neq y\\
      \end{array} \right.
\end{equation}

Portanto apenas traços no mesmo módulo contribuem para o valor de $L$.
Já os $k_i$ e $k_j$  representam o valor total de correlação atribuído a
cada traço, ou, mais claramente, a soma da correlação de cada traço com
todos os outros:

\begin{equation}
   k_i = \sum_{j \neq i} A_{ij}
\end{equation}

E $m$ representa a soma de todos os $k$ ($m=\sum_i k_i$). 
O termo $\frac{k_ik_j}{2m}$ funciona como um tipo de expectativa nula
para a distribuição das correlações entre os traços $i$ e $j$ dado que
cada em deles possui, na matriz $A$, um valor de correlação total
dado por $k_i$ e $k_j$.
Essa escolha para a expectativa nula aparece naturalmente quando
impomos que ela deva ser simétrica entre os traços e que dependa
unicamente de $k_i$ e $k_j$ \citep[Para detalhes veja][]{Newman2006a}.
Portanto, traços no mesmo módulo que possuam correlação maior que a
esperada ao acaso contribuem para aumentar a modularidade $L$, enquanto
traços no mesmo módulo com correlação abaixo da esperada ao acaso
contribuem para diminuir $L$.

Definida a função $L$ de modularidade, temos que encontrar a partição
dos traços (os valores de $g_i$) que maximizem a função $L$. 
Isso é feito com a técnica de Monte Carlo utilizando um algoritmo
do tipo Metropolis-Hasting \citep{Metropolis1953}.
A partição que maximize $L$ é tomada como a estrutura de modularidade do
sistema e o valor máximo de $L$ como um índice de modularidade,
comparável ao AVG-Ratio. 
Neste trabalho, pela estrutura de modularidade ser conhecida, o único propósito 
de usar essa técnica de detecção de módulos é {\it quantificar} a 
modularidade das matrizes sem utilizar o conhecimento prévio de sua
estrutura.
Em situações naturais, onde não conhecemos a estrutura de modularidade,
essa técnica poderia ser usada para inferir a estrutura de modularidade
utilizando apenas da matriz de correlação.

\section{Resumo de Parâmetros}

Para facilitar a leitura do texto, apresentamos aqui um resumo dos
parâmetros relevantes aos modelos.

\begin{table}[htbp]
    \caption[Tabela de abreviações]{Tabela com símbolos e abreviações
    usadas nas simulações}
    \label{tab:exemplo}
    \vspace{1em}
    \centering
    \begin{tabular}{c c c}
        \toprule
        Simbolo     & Interpretação & Modelo\\
        \midrule
        $p$     & número de traços & $M$ e $B$\\
        $m$     & número de alelos aditivos & $M$ e $B$\\
        $Ne$    & tamanho populacional & $M$ e $B$   \\
        $\omega$     & matriz de seleção estabilizadora & $M$ e $B$\\
        $\mu$     & taxa de mutação por alelo aditivo por geração & $M$ e $B$\\
        $\mu_B$     & taxa de mutação por casela da matriz $B$ & $B$\\
        $\Delta_S$     & taxa de mudança do pico adaptativo & $M$ e $B$\\
                       &   por geração de seleção direcional & \\

        \bottomrule
    \end{tabular}
\end{table}
