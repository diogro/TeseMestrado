\pagestyle{empty}
\cleardoublepage
\pagestyle{fancy}
\chapter{Considerações Finais}
\label{cap5}

Neste trabalho procuramos trazer uma abordagem matemática/computacional
tipicamente utilizada em sistemas físicos para o estudo de
modularidade. 
O simples ato de tentar reproduzir o comportamento de um sistema natural
complexo através de interações simples das suas partes nos leva a um melhor
entendimento desse sistema, no sentido de evidenciar quais os
ingredientes necessários para que o sistema se comporte de determinada
maneira, mesmo que esses ingredientes sejam difíceis de ser medidos ou
observados na natureza. 
Além disso, os parâmetros utilizados na criação do modelo tem
interpretações no mundo natural. 
O sistema computacional permite com que estudemos as consequências da
variação desses parâmetros em diversas situações inacessíveis ao
experimento ou à medição retrospectiva. 
Podemos também restringir o espaço de variação desses parâmetros,
facilitando sua estimativa por vias indiretas ou limitando sua
influência em outros processos de interesse. 

Nosso trabalho também se coloca como a primeira tentativa de simulação
de sistemas morfológicos complexos o suficiente, com mais de dois
caracteres, para permitir o surgimento de estruturas de modularidade
variacional complexas. 
Tentativas anteriores se limitam a apenas simular um, ou um número
pequeno, de caracteres. 
Por isso também as exigências computacionais são consideravelmente
maiores no nosso caso. 
Assim, nos limitamos a um tratamento basicamente qualitativo do problema
de evolução da modularidade. 
Nesse contexto, mostramos que seleção direcional correlacionada é
necessária para o surgimento de módulos variacionais. 
Além disso, para a manutenção desses módulos precisamos de seleção
estabilizadora correlacionada, mantendo o padrão de variação
estabelecido pela seleção direcional.
Mostramos ainda que a ação conjunta de seleção direcional e
estabilizadora é suficiente para criar padrões covariação complexos, com
varias classes de correlação diferentes surgindo pela interação entre
os dois tipos de seleção.
Com uma implementação mais sofisticada dos
nossos algoritmos, aliada a um poder computacional maior, acreditamos
que uma extensão quantitativa dos resultados seja não só possível como
bastante proveitosa. 
Pretendemos implementar nosso modelo em linguagens de programação
acessíveis, possibilitando seu uso pela comunidade em geral, com suporte
para paralelização, manipulação de parâmetros com interface amigável e
até criar um projeto integrado de varias possibilidades de simulações
com caracteres quantitativos.

Varias questões interessantes permanecem abertas, e pretendemos
continuar nesta linha de pesquisa.
Já de principio, acreditamos que o modelo nulo, de mutação e deriva, sem
seleção, merece ser extensamente replicado, criando um distribuição nula
para testes estatísticos rigorosos e estudando que tipos de estruturas
variacionais podem surgir por mero acaso.
Seriam essas estruturas estáveis por um tempo apreciável?
Com que frequência o nível de modularidade observado em populações
naturais é observado nesse modelo nulo? 
Além disso, a interação entre tamanho populacional, recombinação,
mutação e seleção se mostrou relevante tanto para o nível de
modularidade observado quanto para a velocidade de estabelecimento do
padrão modular.
Explorar de forma sistemática a relação entre esses parâmetros é um
caminho natural para a continuidade deste trabalho.

Pretendemos também abordar possíveis soluções analíticas para o modelo
proposto. 
Existe algum trabalho em esquemas semelhantes \citep{Wagner1984,
Wagner1989, Jones2007}, mas com abordagens diferentes, não tratando de
modularidade ou regiões fora do equilíbrio, como nas nossas simulações.

Extensões metodológicas também são possíveis.
Em população naturais, a organização genômica em forma de cromossomos
traz uma série de consequências interessantes relacionadas a
desequilíbrio de ligação e recombinação, tema frequentemente
negligenciado mesmo em estudos teóricos complexos \citep{Barton1987,
Turelli1994}.
A ação da seleção cria associações entre regiões
cromossômicas baseadas em características funcionais, aproximando genes
ligados a uma mesma função espacialmente e gerando desequilíbrio de
ligação e herança conjunta de blocos funcionais \citep{Templeton2003}.
Esse processo pode ser importante em definir a velocidade de criação e
reorganização da estruturas genômicas modulares.
A formação dessas associações, que nada mais são que unidades de seleção
acima do nível de genes, é contrabalanceado pela ação da recombinação,
que embaralha as regiões genômicas e quebra ligações formadas pela ação
da seleção.
Incluir estruturas cromossômicas de fato, possibilitando desequilíbrio
de ligação e reestruturações na ligação entre loci traria mais um camada
de complexidade importante na compreensão da evolução da modularidade,
permitindo estudar a dinâmica de formação de unidades de seleção
ao longo da evolução, e sua relação com forças de seleção e recombinação.
Da mesma forma, estudar várias populações simultaneamente, permitindo
migração entre elas, pode trazer resultados interessantes na teoria de
especiação e diversificação de populações.
Atualmente a teoria de genética quantitativa é relativamente pobre para
tratar do problema de fluxo gênico.
Abordagens computacionais podem mudar esse cenário.

Eventos populacionais são reconhecidamente importantes na evolução de
populações naturais, como gargalos, efeito fundador ou mesmo ou
episódios consecutivos de seleção ou mesmo seleção dependente de
frequência podem ter consequências interessantes. 

Outra extensão clara é a de utilizar o nosso modelo para reproduzir
padrões específicos de diversificação observados na natureza. 
Por exemplo, um padrão claro em mamíferos é a diferença entre os níveis
de integração total, característica extremamente variável entre os
grupos, apesar do padrão de integração ser relativamente estável
\cite{Porto2009}. 
Quais cenários evolutivos e tipos de seleção e populações podem levar a
extrema diferença na magnitude de integração observada entre babuínos e grandes
primatas?
A mesma pergunta pode ser feita em relação aos valores altíssimos de
integração presentes em marsupiais. 
Seriam esses valores de altos de integração causados por seleção
direcional correlacionada a todos os caracteres?
Para abordar esses problemas, poderíamos estudar sistemas já com padrões
modulares estabelecidos e simular situações capazes de alterar esses padrões 
nas diferentes populações. 
No momento nossas populações sempre são inciadas totalmente integradas
ou desintegradas, seria interessante estudar a interação da seleção com
padrões já estabelecidos.
Consequências evolutivas claras, na forma de seleção indireta, são
previstas pela teoria de genética quantitativa quantitativa
\citep{Lande1983, Barton1987}, mas essas situações não foram abordadas
computacionalmente em sistemas de muitos caracteres.

