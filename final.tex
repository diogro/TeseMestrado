\pagestyle{empty}
\cleardoublepage
\pagestyle{fancy}
\chapter{Considerações Finais}
\label{cap5}

Neste trabalho procuramos trazer uma abordagem matemática/computacional
tipicamente utilizada em sistemas físicos para o estudo de
modularidade. 
O simples ato de tentar reproduzir o comportamento de um sistema natural
complexo através de interações simples das suas partes nos leva a um melhor
entendimento desse sistema, no sentido de evidenciar quais os
ingredientes necessários para que o sistema se comporte de determinada
maneira, mesmo que esses ingredientes sejam difíceis de ser medidos ou
observados na natureza. 
Além disso, os parâmetros utilizados na criação do modelo tem
interpretações no mundo natural. 
O sistema computacional permite com que estudemos as consequências da
variação desses parâmetros em diversas situações inacessíveis ao
experimento ou à medição retrospectiva. 
Podemos também restringir o espaço de variação desses parâmetros,
facilitando sua estimativa por vias indiretas ou limitando sua
influência em outros processos de interesse. 

Nosso trabalho também se coloca como a primeira tentativa de simulação
de sistemas morfológicos complexos o suficiente para permitir o
surgimento de estruturas de modularidade variacional complexas. 
Tentativas anteriores se limitam a apenas simular um, ou um número
pequeno, de traços. 
Por isso também as exigências computacionais são consideravelmente
maiores no nosso caso. 
Assim, nos limitamos a um tratamento basicamente qualitativo do problema
de evolução da modularidade. 
Nesse contexto, mostramos que seleção direcional correlacionada é
necessária para o surgimento de módulos variacionais. 
Além disso, para a manutenção desses módulos precisamos de seleção
estabilizadora correlacionada, mantendo o padrão de variação
estabelecido pela seleção direcional.
Mostramos ainda que a ação conjunta de seleção direcional e
estabilizadora é suficiente para criar padrões covariação complexos, com
varias classes de correlação diferentes surgindo pela interação entre
os dois tipos de seleção.
Com uma implementação mais sofisticada dos
nossos algoritmos, aliada a um poder computacional maior, acreditamos
que uma extensão quantitativa dos resultados seja não só possível como
bastante proveitosa. 

Outra extensão clara é a de utilizar o nosso modelo para reproduzir
padrões específicos de diversificação observados na natureza. 
Por exemplo, um padrão claro em mamíferos é a diferença entre os níveis
de integração total, característica extremamente variável entre os
grupos, apesar do padrão de integração ser relativamente estável
\cite{Porto2009}. 
Quais cenários evolutivos e tipos de seleção e populações podem levar a
extrema diferença de integração observada entre babuínos e grandes
primatas?
A mesma pergunta pode ser feita em relação aos valores altíssimos de
integração presentes em marsupiais. 
Estes são apenas exemplos de problemas em aberto que podem ser abordados
por meio de simulações semelhantes às desenvolvidas neste trabalho. 
Para abordar esses problemas, poderíamos estudar sistemas já com padrões
modulares estabelecidos e simular situações capazes de alterar esses padrões 
nas diferentes populações. 
No momentos nossas populações sempre são inciadas totalmente integradas
ou desintegradas, seria interessante estudar a interação da seleção com
padrões já estabelecidos.
Consequências evolutivas claras, na forma de seleção indireta, são
previstas pela teoria de genética quantitativa quantitativa
\citep{Lande1983, Barton1987}, mas essas situações não foram abordadas
computacionalmente em sistemas de muitos traços.

Ainda existem muitas extensões possíveis ao nosso modelo. 
Eventos populacionais, como gargalos ou episódios consecutivos de
seleção ou mesmo seleção dependente de frequência podem ter
consequências interessantes. 
O estudos de especiação ou diferenciação entre populações filhas, com
diversas filogenias também são possíveis. 
Inclusão de interação entre populações, como migração ou competição por
recursos ou qualquer outro tipo de interação ecológica pode ser incluído
no contexto de simulação de traços quantitativos. 
