%%% Pacotes utilizados %%%
%DIF PREAMBLE EXTENSION ADDED BY LATEXDIFF
%DIF UNDERLINE PREAMBLE %DIF PREAMBLE
\RequirePackage[normalem]{ulem} %DIF PREAMBLE
\RequirePackage{color}\definecolor{RED}{rgb}{1,0,0}\definecolor{BLUE}{rgb}{0,0,1} %DIF PREAMBLE
\providecommand{\DIFadd}[1]{{\protect\color{blue}\uwave{#1}}} %DIF PREAMBLE
\providecommand{\DIFdel}[1]{{\protect\color{red}\sout{#1}}}                      %DIF PREAMBLE
%DIF SAFE PREAMBLE %DIF PREAMBLE
\providecommand{\DIFaddbegin}{} %DIF PREAMBLE
\providecommand{\DIFaddend}{} %DIF PREAMBLE
\providecommand{\DIFdelbegin}{} %DIF PREAMBLE
\providecommand{\DIFdelend}{} %DIF PREAMBLE
%DIF FLOATSAFE PREAMBLE %DIF PREAMBLE
\providecommand{\DIFaddFL}[1]{\DIFadd{#1}} %DIF PREAMBLE
\providecommand{\DIFdelFL}[1]{\DIFdel{#1}} %DIF PREAMBLE
\providecommand{\DIFaddbeginFL}{} %DIF PREAMBLE
\providecommand{\DIFaddendFL}{} %DIF PREAMBLE
\providecommand{\DIFdelbeginFL}{} %DIF PREAMBLE
\providecommand{\DIFdelendFL}{} %DIF PREAMBLE
%DIF END PREAMBLE EXTENSION ADDED BY LATEXDIFF
%% Codificação e formatação básica do LaTeX
% Suporte para português (hifenação e caracteres especiais)
\usepackage[english,brazilian]{babel}

% Codificação do arquivo
\usepackage[utf8]{inputenc}

% Mapear caracteres especiais no PDF
\usepackage{cmap}

% Codificação da fonte
\usepackage[T1]{fontenc}

% Essencial para colocar funções e outros símbolos matemáticos
\usepackage{amsmath,amssymb,amsfonts,textcomp}
\DeclareUnicodeCharacter{00A0}{~}
%% Layout
% Customização do layout da página, margens espelhadas
\usepackage[twoside]{geometry}
% Aumenta as margens internas para espiral
\geometry{bindingoffset=10pt}
% Só pra ajustar o layout
\setlength{\marginparwidth}{90pt}
%\usepackage{layout}

% Para definir espaçamento entre as linhas
\usepackage{setspace}

% Espaçamento do texto para o frame
\setlength{\fboxsep}{1em}

% Faz com que as margens tenham o mesmo tamanho horizontalmente
%\geometry{hcentering}

%% Elementos Gráficos
% Para incluir figuras (pacote extendido)
\usepackage[]{graphicx}

% Suporte a cores
\usepackage{color}

% Criar figura dividida em subfiguras
\usepackage{subfig}
\captionsetup[subfigure]{style=default, margin=0pt, parskip=0pt, hangindent=0pt, indention=0pt, singlelinecheck=true, labelformat=parens, labelsep=space}

% Caso queira guardar as figuras em uma pasta separada
% (descomente e) defina o caminho para o diretório:
%\graphicspath{{./figuras/}}

% Customizar as legendas de figuras e tabelas
\usepackage{caption}

% Criar ambientes com 2 ou mais colunas
\usepackage{multicol}

% Ative o comando abaixo se quiser colocar figuras de fundo (e.g., capa)
%\usepackage{wallpaper}
% Exemplo para inserir a figura na capa está no arquivo pre.tex (linha 7)
% Ajuste da posição da figura no eixo Y
%\addtolength{\wpYoffset}{-140pt}
% Ajuste da posição da figura no eixo X
%\addtolength{\wpXoffset}{36pt}

%% Tabelas
% Elementos extras para formatação de tabelas
\usepackage{array}

% Tabelas com qualidade de publicação
\usepackage{booktabs}

% Para criar tabelas maiores que uma página
\usepackage{longtable}

% adicionar tabelas e figuras como landscape
\usepackage{lscape}

%% Lista de Abreviações
% Cria lista de abreviações
\usepackage[notintoc,portuguese]{nomencl}
\makenomenclature

%% Notas de rodapé
% Lidar com notas de rodapé em diversas situações
\usepackage{footnote}

% Notas criadas nas tabelas ficam no fim das tabelas
\makesavenoteenv{tabular}

%% Links dinâmicos
% Suporte para hipertexto, links para referências e figuras
\usepackage{hyperref}
% Configurações dos links e metatags do PDF a ser gerado
\hypersetup{colorlinks=true, linkcolor=blue, citecolor=blue, filecolor=blue, pagecolor=blue, urlcolor=green,
            pdfauthor={Diogo Amaral Rebouças Melo},
            pdftitle={Evolução Morfológica e Modularidade},
            pdfsubject={Assunto do Projeto},
            pdfkeywords={genética quantitativa, modularidade, matriz G, simulação computacional, seleção direcional},
            pdfproducer={Latex},
            pdfcreator={pdflatex}}

% Conta o número de páginas
\usepackage{lastpage}

%% Referências bibliográficas e afins
% Formatar as citações no texto e a lista de referências
\usepackage{natbib}

% Adicionar bibliografia, índice e conteúdo na Tabela de conteúdo
% Não inclui lista de tabelas e figuras no índice
\usepackage[nottoc,notlof,notlot]{tocbibind}

%% Pontuação e unidades
% Posicionar inteligentemente a vírgula como separador decimal
\usepackage{icomma}

% Formatar as unidades com as distâncias corretas
\usepackage[tight]{units}

%% Cabeçalho e rodapé
% Controlar os cabeçalhos e rodapés
\usepackage{fancyhdr}
% Usar os estilos do pacote fancyhdr
\pagestyle{fancy}
\fancypagestyle{plain}{\fancyhf{}}
% Limpar os campos do cabeçalho atual
\fancyhead{}
% Número da página do lado esquerdo [L] nas páginas ímpares [O] e do lado direito [R] nas páginas pares [E]
\fancyhead[LO,RE]{\thepage}
% Nome da seção do lado direito em páginas ímpares
\fancyhead[RO]{\nouppercase{\rightmark}}
% Nome do capítulo do lado esquerdo em páginas pares
\fancyhead[LE]{\nouppercase{\leftmark}}
% Limpar os campos do rodapé
\fancyfoot{}
% Omitir linha de separação entre cabeçalho e conteúdo
\renewcommand{\headrulewidth}{0pt}
% Omitir linha de separação entre rodapé e conteúdo
\renewcommand{\footrulewidth}{0pt}
% Altura do cabeçalho
\headheight 13.6pt

%% Inserir comentários no texto
% Marcar mudanças e fazer comentários
%\usepackage[margins]{trackchanges}
% Iniciais do autor
%\renewcommand{\initialsTwo}{bcv}
% Notas na margem interna
%\reversemarginpar

%% Comandos customizados

% Espécie e abreviação
\newcommand{\tracos}{caracteres}
\newcommand{\traco}{caráter}

% Título do projeto
\newcommand{\titulo}{Evolução Morfológica e Modularidade}
\newcommand{\nomedoaluno}{Diogo Amaral Rebouças Melo}

%% Pacotes não implementados
% Para não sobrar espaços em branco estranhos
%\widowpenalty=1000
%\clubpenalty=1000
