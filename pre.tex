% Capa
\begin{titlepage}
% Se quiser uma figura de fundo na capa ative o pacote wallpaper
% e descomente a linha abaixo.
% \ThisCenterWallPaper{0.8}{nomedafigura}

\begin{center}
{\LARGE \nomedoaluno}
\par
\vspace{200pt}
{\Huge \titulo}
\par
\vfill
\textbf{{\large São Paulo}\\
{\large 2012}}
\end{center}
\end{titlepage}

% A partir daqui páginas sem cabeçalho
\pagestyle{empty}
% Faz com que a página seguinte sempre seja ímpar (insere pg em branco)
\cleardoublepage

% Números das páginas em algarismos romanos
\pagenumbering{roman}

% Página de Rosto
\begin{center}
{\LARGE \nomedoaluno}
\par
\vspace{200pt}
{\Huge \titulo}
\end{center}
\par
\vspace{90pt}
\hspace*{175pt}\parbox{7.6cm}{{\large Dissertação apresentada ao Instituto de Biociências da Universidade de São Paulo, para a obtenção de Título de Mestre em Ciências, na Área de Biologia Evolutiva.}}

\par
\vspace{1em}
\hspace*{175pt}\parbox{7.6cm}{{\large Orientador: Gabriel Marroig}}

\par
\vfill
\begin{center}
\textbf{{\large São Paulo}\\
{\large 2012}}
\end{center}

\newpage

% Ficha Catalográfica
\hspace{8em}\fbox{\begin{minipage}{10cm}
Diogo Amaral Rebouças Melo

\hspace{2em}\titulo

\hspace{2em}\pageref{LastPage} páginas

\hspace{2em}Dissertação (Mestrado) - Instituto de Biociências da Universidade de São Paulo. Departamento de Genética e Biologia Evolutiva.

\begin{enumerate}
   \item Genética quantitativa;
   \item Modularidade;
   \item Matriz G;
   \item Simulação computacional;
   \item Seleção direcional.
\end{enumerate}
I. Universidade de São Paulo. Instituto de Biociências. Departamento de Genética e Biologia Evolutiva.

\end{minipage}}
\par
\vspace{2em}
\begin{center}
{\LARGE\textbf{Comissão Julgadora:}}

\par
\vspace{10em}
\begin{tabular*}{\textwidth}{@{\extracolsep{\fill}}l l}
\rule{16em}{1px} 	& \rule{16em}{1px} \\
Prof. Dr. 		& Prof. Dr. \\
Nome			& Nome
\end{tabular*}

\par
\vspace{10em}

\parbox{16em}{\rule{16em}{1px} \\
Prof. Dr. \\
Gabriel Marroig}
\end{center}

\newpage

% Dedicatória
% Posição do texto na página
\vspace*{0.75\textheight}
\begin{flushright}
  \emph{Dedicatória...}
\end{flushright}

\newpage

% Epígrafe
\vspace*{0.4\textheight}
\noindent{\LARGE\textbf{Epígrafe}}
% Tudo que você escreve no verbatim é renderizado literalmente (comandos não são interpretados e os espaços são respeitados)
\begin{verbatim}
No man is an island entire of itself; every man 
is a piece of the continent, a part of the main; 
if a clod be washed away by the sea, Europe 
is the less, as well as if a promontory were, as 
well as a manor of thy friends or of thine 
own were; any man's death diminishes me, 
because I am involved in mankind. 
And therefore never send to know for whom 
the bell tolls; it tolls for thee.
\end{verbatim}
\begin{flushright}
John Donne
\end{flushright}

\newpage

% Agradecimentos

% Espaçamento duplo
\doublespacing

\noindent{\LARGE\textbf{Agradecimentos}}

Agradeço ao meu orientador, aos meus colaboradores, aos técnicos, à seção administrativa, à fundação que liberou verba para minhas pesquisas, aos meus amigos, à minha família e ao meu grande amor.

\newpage

\vspace*{10pt}
% Abstract
\begin{center}
  \emph{\begin{large}Resumo\end{large}}\label{resumo}
\vspace{2pt}
\end{center}
\noindent
Sistemas morfológicos quantitativos são descritos por medidas
contínuas.
A relação genética entre essas caractéristicas dos indivíduos é
representada pela matriz de covariância genética aditiva, a matriz G.
Entender a evolução da matriz G, portanto, é de suma importância para
compreender os padrões de diversificação encontrados na natureza.
Neste trabalho estudamos modelos computacionais para a evolução de
traços contínuos em populações naturais, sujeitas a variados tipos de
seleção e condições internas, focando no problema da evolução dos
padrões de integração e modularidade nessas populações.
Testamos dois modelos com diferentes combinações de parâmetros em sua
capacidade de reproduzir e elucidar padrões naturais.
Seleção direcional correlacionada se mostrou uma força importante na criação desses
padrões de covariação e a seleção estabilizadora correlacionada se mostrou fundamental
para a manutenção desses padrões.
\par
\vspace{1em}
\noindent\textbf{Palavras-chave:} genética quantitativa, modularidade, matriz G, simulação computacional, seleção direcional
\newpage

% Criei a página do abstract na mão, por isso tem bem mais comandos do que o resumo acima, apesar de serem idênticas.
\vspace*{10pt}
% Abstract
\begin{center}
  \emph{\begin{large}Abstract\end{large}}\label{abstract}
\vspace{2pt}
\end{center}

% Selecionar a linguagem acerta os padrões de hifenação diferentes entre inglês e português.
\selectlanguage{english}
\noindent
Quantitative morphological systems are described by continuous measurements. 
The genetic relation between these characteristics of the individuals is
represented by the genetic additive co-variance matrix, the G matrix.
Understanding the evolution of the G matrix is, therefore, of
paramount importance for proper interpretation of the patterns of
diversification we observe in nature.
In this work we study computational models for the evolution of
quantitative traits in natural populations, subject to different natural
selection and internal conditions, focusing on the problem  of the
evolution of the pattern of morphological integration and modularity. 
We test two models with different sets of parameters in their ability to
reproduce and elucidate natural patterns. 
Directional correlated selection was necessary for the shaping of the
patterns of morphological integration,
and correlated stabilizing selection was fundamental to
the maintenance of these patterns.
\par
\vspace{1em}
\noindent\textbf{Keywords:} quantitative genetics, modularity, G matrix,
computational simulation, directional selection

% Voltando ao português...
\selectlanguage{brazilian}

\newpage

% Lista de figuras
\listoffigures

% Lista de tabelas
\listoftables

% Abreviações
% Para imprimir as abreviações siga as instruções em 
% http://code.google.com/p/mestre-em-latex/wiki/ListaDeAbreviaturas
\printnomenclature

% Índice
\tableofcontents
