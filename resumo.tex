% resumo.tex
% (c) 2012 Diogo Amaral Reboucas Melo <diogro@gmail.com>
%
\documentclass[twoside,a4paper,11pt]{report}

% Pacotes e comandos customizados
\include{meta}

\begin{document}
\begin{center}
  \emph{\begin{large}Resumo\end{large}}\label{resumo}
\vspace{2pt}
\end{center}
\noindent
Sistemas morfológicos quantitativos são descritos por medidas
contínuas.
A relação genética entre essas características dos indivíduos é
representada pela matriz de covariância genética aditiva, a matriz G.
Entender a evolução da matriz G, portanto, é de suma importância para
compreender os padrões de diversificação encontrados na natureza.
Neste trabalho estudamos modelos computacionais para a evolução de
traços contínuos em populações naturais, sujeitas a variados tipos de
seleção e condições internas, focando no problema da evolução dos
padrões de integração e modularidade nessas populações.
Testamos dois modelos com diferentes combinações de parâmetros em sua
capacidade de reproduzir e elucidar padrões naturais.
Seleção direcional correlacionada se mostrou uma força importante na criação desses
padrões de covariação e a seleção estabilizadora correlacionada se mostrou fundamental
para a manutenção desses padrões.
\par
\vspace{1em}
\noindent\textbf{Palavras-chave:} genética quantitativa, modularidade, matriz G, simulação computacional, seleção direcional

% Criei a página do abstract na mão, por isso tem bem mais comandos do que o resumo acima, apesar de serem idênticas.
\vspace*{10pt}
% Abstract
\begin{center}
  \emph{\begin{large}Abstract\end{large}}\label{abstract}
\vspace{2pt}
\end{center}

% Selecionar a linguagem acerta os padrões de hifenação diferentes entre inglês e português.
\selectlanguage{english}
\noindent
Quantitative morphological systems are described by continuous measurements. 
The genetic relation between these characteristics of the individuals is
represented by the genetic additive co-variance matrix, the G matrix.
Understanding the evolution of the G matrix is, therefore, of
paramount importance for proper interpretation of the patterns of
diversification we observe in nature.
In this work we study computational models for the evolution of
quantitative traits in natural populations, subject to different natural
selection and internal conditions, focusing on the problem  of the
evolution of the pattern of morphological integration and modularity. 
We test two models with different sets of parameters in their ability to
reproduce and elucidate natural patterns. 
Directional correlated selection was necessary for the shaping of the
patterns of morphological integration,
and correlated stabilizing selection was fundamental to
the maintenance of these patterns.
\par
\vspace{1em}
\noindent\textbf{Keywords:} quantitative genetics, modularity, G matrix,
computational simulation, directional selection

% Voltando ao português...
\selectlanguage{brazilian}


\flushright{Diogo Melo $\cdot$ 25 de julho de 2012}

\end{document}
