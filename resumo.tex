% resumo.tex
% (c) 2012 Diogo Amaral Reboucas Melo <diogro@gmail.com>
%
\documentclass[twoside,a4paper,11pt]{report}

% Pacotes e comandos customizados
%%% Pacotes utilizados %%%

%% Codificação e formatação básica do LaTeX
% Suporte para português (hifenação e caracteres especiais)
\usepackage[english,brazilian]{babel}

% Codificação do arquivo
\usepackage[utf8]{inputenc}

% Mapear caracteres especiais no PDF
\usepackage{cmap}

% Codificação da fonte
\usepackage[T1]{fontenc}

% Essencial para colocar funções e outros símbolos matemáticos
\usepackage{amsmath,amssymb,amsfonts,textcomp}

%% Layout
% Customização do layout da página, margens espelhadas
\usepackage[twoside]{geometry}
% Aumenta as margens internas para espiral
\geometry{bindingoffset=10pt}
% Só pra ajustar o layout
\setlength{\marginparwidth}{90pt}
%\usepackage{layout}

% Para definir espaçamento entre as linhas
\usepackage{setspace}

% Espaçamento do texto para o frame
\setlength{\fboxsep}{1em}

% Faz com que as margens tenham o mesmo tamanho horizontalmente
%\geometry{hcentering}

%% Elementos Gráficos
% Para incluir figuras (pacote extendido)
\usepackage[]{graphicx}

% Suporte a cores
\usepackage{color}

% Criar figura dividida em subfiguras
\usepackage{subfig}
\captionsetup[subfigure]{style=default, margin=0pt, parskip=0pt, hangindent=0pt, indention=0pt, singlelinecheck=true, labelformat=parens, labelsep=space}

% Caso queira guardar as figuras em uma pasta separada
% (descomente e) defina o caminho para o diretório:
%\graphicspath{{./figuras/}}

% Customizar as legendas de figuras e tabelas
\usepackage{caption}

% Criar ambientes com 2 ou mais colunas
\usepackage{multicol}

% Ative o comando abaixo se quiser colocar figuras de fundo (e.g., capa)
%\usepackage{wallpaper}
% Exemplo para inserir a figura na capa está no arquivo pre.tex (linha 7)
% Ajuste da posição da figura no eixo Y
%\addtolength{\wpYoffset}{-140pt}
% Ajuste da posição da figura no eixo X
%\addtolength{\wpXoffset}{36pt}

%% Tabelas
% Elementos extras para formatação de tabelas
\usepackage{array}

% Tabelas com qualidade de publicação
\usepackage{booktabs}

% Para criar tabelas maiores que uma página
\usepackage{longtable}

% adicionar tabelas e figuras como landscape
\usepackage{lscape}

%% Lista de Abreviações
% Cria lista de abreviações
\usepackage[notintoc,portuguese]{nomencl}
\makenomenclature

%% Notas de rodapé
% Lidar com notas de rodapé em diversas situações
\usepackage{footnote}

% Notas criadas nas tabelas ficam no fim das tabelas
\makesavenoteenv{tabular}

%% Links dinâmicos
% Suporte para hipertexto, links para referências e figuras
\usepackage{hyperref}
% Configurações dos links e metatags do PDF a ser gerado
\hypersetup{colorlinks=true, linkcolor=blue, citecolor=blue, filecolor=blue, pagecolor=blue, urlcolor=green,
            pdfauthor={Nome do Autor},
            pdftitle={Título do Projeto},
            pdfsubject={Assunto do Projeto},
            pdfkeywords={palavra-chave, palavra-chave, palavra-chave},
            pdfproducer={Latex},
            pdfcreator={pdflatex}}

% Conta o número de páginas
\usepackage{lastpage}

%% Referências bibliográficas e afins
% Formatar as citações no texto e a lista de referências
\usepackage{natbib}

% Adicionar bibliografia, índice e conteúdo na Tabela de conteúdo
% Não inclui lista de tabelas e figuras no índice
\usepackage[nottoc,notlof,notlot]{tocbibind}

%% Pontuação e unidades
% Posicionar inteligentemente a vírgula como separador decimal
\usepackage{icomma}

% Formatar as unidades com as distâncias corretas
\usepackage[tight]{units}

%% Cabeçalho e rodapé
% Controlar os cabeçalhos e rodapés
\usepackage{fancyhdr}
% Usar os estilos do pacote fancyhdr
\pagestyle{fancy}
\fancypagestyle{plain}{\fancyhf{}}
% Limpar os campos do cabeçalho atual
\fancyhead{}
% Número da página do lado esquerdo [L] nas páginas ímpares [O] e do lado direito [R] nas páginas pares [E]
\fancyhead[LO,RE]{\thepage}
% Nome da seção do lado direito em páginas ímpares
\fancyhead[RO]{\nouppercase{\rightmark}}
% Nome do capítulo do lado esquerdo em páginas pares
\fancyhead[LE]{\nouppercase{\leftmark}}
% Limpar os campos do rodapé
\fancyfoot{}
% Omitir linha de separação entre cabeçalho e conteúdo
\renewcommand{\headrulewidth}{0pt}
% Omitir linha de separação entre rodapé e conteúdo
\renewcommand{\footrulewidth}{0pt}
% Altura do cabeçalho
\headheight 13.6pt

%% Inserir comentários no texto
% Marcar mudanças e fazer comentários
%\usepackage[margins]{trackchanges}
% Iniciais do autor
%\renewcommand{\initialsTwo}{bcv}
% Notas na margem interna
%\reversemarginpar

%% Comandos customizados

% Espécie e abreviação
\newcommand{\subde}{\emph{Clypeaster subdepressus}}
\newcommand{\subsus}{\emph{C.~subdepressus}}

% Título do projeto
\newcommand{\titulo}{Título original do projeto}
\newcommand{\nomedoaluno}{Nome Completo do Aluno}

%% Pacotes não implementados
% Para não sobrar espaços em branco estranhos
%\widowpenalty=1000
%\clubpenalty=1000


\begin{document}
\begin{center}
  \emph{\begin{large}Resumo\end{large}}\label{resumo}
\vspace{2pt}
\end{center}
\noindent
Sistemas morfológicos quantitativos são descritos por medidas
contínuas.
A relação genética entre essas características dos indivíduos é
representada pela matriz de covariância genética aditiva, a matriz G.
Entender a evolução da matriz G, portanto, é de suma importância para
compreender os padrões de diversificação encontrados na natureza.
Neste trabalho estudamos modelos computacionais para a evolução de
traços contínuos em populações naturais, sujeitas a variados tipos de
seleção e condições internas, focando no problema da evolução dos
padrões de integração e modularidade nessas populações.
Testamos dois modelos com diferentes combinações de parâmetros em sua
capacidade de reproduzir e elucidar padrões naturais.
Seleção direcional correlacionada se mostrou uma força importante na criação desses
padrões de covariação e a seleção estabilizadora correlacionada se mostrou fundamental
para a manutenção desses padrões.
\par
\vspace{1em}
\noindent\textbf{Palavras-chave:} genética quantitativa, modularidade, matriz G, simulação computacional, seleção direcional

% Criei a página do abstract na mão, por isso tem bem mais comandos do que o resumo acima, apesar de serem idênticas.
\vspace*{10pt}
% Abstract
\begin{center}
  \emph{\begin{large}Abstract\end{large}}\label{abstract}
\vspace{2pt}
\end{center}

% Selecionar a linguagem acerta os padrões de hifenação diferentes entre inglês e português.
\selectlanguage{english}
\noindent
Quantitative morphological systems are described by continuous measurements. 
The genetic relation between these characteristics of the individuals is
represented by the genetic additive co-variance matrix, the G matrix.
Understanding the evolution of the G matrix is, therefore, of
paramount importance for proper interpretation of the patterns of
diversification we observe in nature.
In this work we study computational models for the evolution of
quantitative traits in natural populations, subject to different natural
selection and internal conditions, focusing on the problem  of the
evolution of the pattern of morphological integration and modularity. 
We test two models with different sets of parameters in their ability to
reproduce and elucidate natural patterns. 
Directional correlated selection was necessary for the shaping of the
patterns of morphological integration,
and correlated stabilizing selection was fundamental to
the maintenance of these patterns.
\par
\vspace{1em}
\noindent\textbf{Keywords:} quantitative genetics, modularity, G matrix,
computational simulation, directional selection

% Voltando ao português...
\selectlanguage{brazilian}


\flushright{Diogo Melo $\cdot$ 25 de julho de 2012}

\end{document}
